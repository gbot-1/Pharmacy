
%%%% RAPPORT FUSION %%%%%%%%%

% il faut une introduction differente des données : fusion avec une phie speciale {{ Name_phamra_fusion }} donc entre son APB pour avoir les coordonnes x et y de le phie de fusion ainsi que son adresse 
% et meme dans le texte "demande de fusion de la phie A avec la B à l'adresse B %
%
% MANQUE TABLE PEOPLE 50 PERCENT


\documentclass[11pt]{article}
\usepackage[a4paper, margin=1in]{geometry}
\usepackage[french]{babel}
\usepackage{graphicx}
\usepackage{tabularx}
\usepackage{float}
\usepackage{hyperref}
\usepackage{pdfpages}

\usepackage{ifthen}
\newboolean{boolnearest1000}
\setboolean{boolnearest1000}{{{ Bool_nearest_1000 }}}

\begin{document}
\pagestyle{empty}

\begin{titlepage}
\begin{flushright}
    \textit{Roselies, le \today }
\end{flushright}

\begin{minipage}{0.5\textwidth}
\begin{flushleft}
    \textbf{\LARGE MATHIEU Raphael} \\
    Ingénieur Géomètre GEO21/1547 \\
    \vspace{3mm}
    Rue J. Wauters 52\\
    6250 Roselies \\
    \vspace{3mm}
    Email: \href{mailto:mathieu.geometre@gmail.com}{mathieu.geometre@gmail.com}\\
    GSM: \href{tel:+32471784815}{0471/78.48.15} \\
\end{flushleft}
\end{minipage}
\hfill
\begin{minipage}{0.5\textwidth}
\begin{flushright}
    \includegraphics[scale=0.4]{"C:/Users/no-st/Code_Pharmarcie/Codes/LogoGeometre.png"}  
\end{flushright}
\end{minipage}

\vspace{10mm}

\hrule

\begin{center}
    \huge\textbf{Demande de fusion de la pharmacie}\\
    \vspace{2mm}
    \huge\textbf{{{ Name_pharma }} (APB {{ Id_pharma }})} avec la {{ Name_pharma_fusion }} (APB {{ Id_pharma_fusion }})\\
\end{center}
\vspace{30mm}
\includegraphics[width=\textwidth]{{{ Map_new_implentation }}}

\end{titlepage}


\clearpage% Body
\begin{center}
    \section*{\Huge Demande de fusion}
\end{center}

\vspace{10mm}

\noindent Rapport de fusion de la pharmacie {{ Name_pharma }} (N° APB {{ Id_pharma }}) {{ Old_adress }}- {{ Old_postcode }} {{ Old_town }} 
avec la pharmacie {{ Name_pharma_fusion }} {{ New_adress }} - {{ New_postcode }} {{ New_town }}.

\vspace{25mm}

\section*{Coordonnées Lambert 2008}
\hfill
\begin{minipage}{0.5\linewidth}
    Adresse de fusion:\\
    X : {{ New_X }} m\\
    Y : {{ New_Y }} m\\
\end{minipage}

\vspace{10mm}



\section*{Distance entre les deux adresses}
\noindent Distance totale entre les deux adresses par la voie publique : {{ Distance_road }} m\\

\noindent Distance "à vol d'oiseau" : {{ Distance_fly }} m\\

\vfill
\noindent La fusion se faisant au sein de la commune de {{ Old_town }}, la condition de transférer dans la même 
commune ou dans une commune limitrophe est respectée.
\vfill

\vfill

\ifthenelse{\boolean{boolnearest1000}}{
    \noindent \textbf{\large La pharmacie {{ Name_pharma_near }} (APB {{ Id_pharma_near }}) se situe à {{ Distance_road_near }} m 
    de la pharamcie {{ Name_pharma }}, fermée à l'issue de cette fusion. Or, selon l’AR du 16 janvier 2022, Chapitre II, section 2, 
    Art. 9, alinéa 1e, 3°, qui renvoi à l’Art.8, §1e, alinéa 1e, 2°, a), de cet AR, l’officine se trouve à moins de 1000m, il n’y a 
    donc pas lieu de calculer une zone d’influence géographique, définissant la zone d’influence démographique.}
}{ 
    \noindent \textbf{\large La pharmacie {{ Name_pharma_near }} (APB {{ Id_pharma_near }}) se situe à {{ Distance_road_near }} m 
    de la pharamcie {{ Name_pharma }}, fermée à l'issue de cette fusion.}
}


\vspace{15mm}
\noindent Rapport rédigé par MATHIEU Raphaël, Ingénieur géomètre, légalement admis et assermenté en cette qualité par le Tribunal de première Instance séant à Namur, inscrit au Conseil Fédéral des géomètres-experts sous le numéro GEO/21/1547.
\vspace{10mm}
\begin{flushright}
    \textit{Rapport établi à Aiseau-Presles le \today }
\end{flushright}

\clearpage
\include{table_old_implentation}

\includepdf[fitpaper,rotateoversize]{{{ Map_new_implentation }}}
 
\ifthenelse{\NOT\boolean{boolnearest1000}}{
  \includepdf[fitpaper,rotateoversize]{{{ Map_all_itinerary_new }}}
}{}

\includepdf[fitpaper,rotateoversize]{{{ Map_all_itinerary_old }}}

\end{document}